\chapter{Introduction}
\label{chapter:intro}

Social media platforms are no longer an emerging field, they have become well established and millions of messages are exchanged daily. Services try to keep up with this trend by promoting popular content, either by number of clicks, views, favorites or other metrics. In this paper, we present a study on the clustering of messages from the Twitter platform, also known as "tweets". Inspired by the website's "Trends" which presents the most popular subjects either worldwide or in a region, the aim of our paper is to explore more of the popular subjects and cluster the conversations on more than just a keyword. The purpose of the clustering is to offer an in depth view of the conversation on a particular topic where peoples opinions may differ greatly.
\newline
The process is separated into three parts: data collection via the Twitter public API, message annotation with part-of-speech tags and message clustering. The messages provided by the website's API already provide a filtering option, you can specify keywords that you want to be part of the messages you get back. This might provide some indication of the conversation topic but getting an overview of the different conversations on the same subject is not a trivial task because messages have no obvious order.
\newline
By clustering the conversations together and offering an interface in which to explore the information, it is possible to view multiple points of view on the same subject and get more information than a popular topic might provide.

\section{Project Description}
\label{sec:proj}

In this paper we first discuss related work in the field. The popularity of the medium and the large quantity of messages have made this subject a popular topic of research. We then continue by giving a detailed explanation of the clustering process, the system architecture and the parsing and message aggregation. The next section will go into details regarding testing and the results reached and the last part will offer suggestions for further improvements and research.

\subsection{Project Scope}
\label{sub-sec:proj-scope}

We will first offer a broader explanation of the problem we wish to solve.
\newline
Twitter\footnote{https://support.twitter.com/articles/49309-using-hashtags-on-twitter} is an online micro blogging platform and social networking website. Users communicate through short 140 character messages called \"tweets\". To ease communication people use \"mentions\", the \"@\" character followed by a persons name. This is a way to involve another account into the conversation. Another feature of the conversation is the \"hashtag\", the \"#\" sign followed by a word this is used to highlight key parts in the message, either a feeling or subject. Popular such hashtags are included in the Trending Topics. These are popular subjects automatically generated from conversations taking place worldwide or in a certain region. This is why people often times include hashtags in order to associate their message with a popular topic. Users can also favorite a tweet, and "retweet" it, meaning they share it with the people that follow them while still attributing the message to the original author. These two metrics contribute to the overall popularity of a tweet.
\newline
Twitter is an interesting platform for research because of the large number of messages exchanged daily. There are 500 million tweets sent daily by its 302 million monthly active users\footnote{https://about.twitter.com/company} with a record of 143,199 tweets per second\footnote{https://blog.twitter.com/2013/new-tweets-per-second-record-and-how}. People usually turn to Twitter during major natural events, sporting events, award ceremonies and so on. With such a high amount of information coming in every second it is almost impossible to keep track of everything that is discussed.

Ce nu ar trebui sa lipseasca: 
- cum ai gandit rezolvarea problemei
- care este arhitectura aplicatiei
- ce ai facut tu ca implementare
- ce metoda de testare ai folosit
- care sunt rezultatele obtinute

Chestiile astea ar trebui sa fie in jur de 30 de pagini. In plus, inainte ar trebui sa ai o introducere in care sa pui:
- descrierea problemei 1-2 pagini
- state-of-the-art (ce s-a mai facut similar) -3-4 pagini
- scurta descriere Twitter (jumatate de pagina cred ca ar fi suficient) + - ce instrumente externe ai folosit (maxim 3-4 pagini)
% my work

The scope of the project \textbf{\project} is to provide close to realtime 
clustering of conversations that take place in the online medium. My choice for
a social network is Twitter. Twitter has around 302 million active users (May 2015)
\footnote{\url{https://about.twitter.com/company}} who send 500 million tweets
each day mostly from their mobile phones. A tweet is a 140 character long message
and because of this conversations are hard to keep track of and provide little to
no context on their subject. This makes them an excellent candidate for a clustering application like \textbf{\project} which aims to provide an overview for  
conversations spanning over all the topics the user of the application provided.

% fillers

This thesis presents the \textbf{\project}.

This is an example of a footnote \footnote{\url{www.google.com}}. You can see here a reference to \labelindexref{Section}{sub-sec:proj-objectives}.

Here we have defined the CS abbreviation.\abbrev{CS}{Computer Science} and the UPB abbreviation.\abbrev{UPB}{University Politehnica of Bucharest}

The main scope of this project is to qualify xLuna for use in critical systems.


Lorem ipsum dolor sit amet, consectetur adipiscing elit. Aenean aliquam lectus vel orci malesuada accumsan. Sed lacinia egestas tortor, eget tristiqu dolor congue sit amet. Curabitur ut nisl a nisi consequat mollis sit amet quis nisl. Vestibulum hendrerit velit at odio sodales pretium. Nam quis tortor sed ante varius sodales. Etiam lacus arcu, placerat sed laoreet a, facilisis sed nunc. Nam gravida fringilla ligula, eu congue lorem feugiat eu.

Lorem ipsum dolor sit amet, consectetur adipiscing elit. Aenean aliquam lectus vel orci malesuada accumsan. Sed lacinia egestas tortor, eget tristiqu dolor congue sit amet. Curabitur ut nisl a nisi consequat mollis sit amet quis nisl. Vestibulum hendrerit velit at odio sodales pretium. Nam quis tortor sed ante varius sodales. Etiam lacus arcu, placerat sed laoreet a, facilisis sed nunc. Nam gravida fringilla ligula, eu congue lorem feugiat eu.


\subsection{Project Objectives}
\label{sub-sec:proj-objectives}

We have now included \labelindexref{Figure}{img:report-framework}.

\fig[scale=0.5]{src/img/reporting-framework.pdf}{img:report-framework}{Reporting Framework}


Lorem ipsum dolor sit amet, consectetur adipiscing elit. Aenean aliquam lectus vel orci malesuada accumsan. Sed lacinia egestas tortor, eget tristiqu dolor congue sit amet. Curabitur ut nisl a nisi consequat mollis sit amet quis nisl. Vestibulum hendrerit velit at odio sodales pretium. Nam quis tortor sed ante varius sodales. Etiam lacus arcu, placerat sed laoreet a, facilisis sed nunc. Nam gravida fringilla ligula, eu congue lorem feugiat eu.

We can also have citations like \cite{iso-odf}.

\subsection{Related Work}

Lorem ipsum dolor sit amet, consectetur adipiscing elit. Aenean aliquam lectus vel orci malesuada accumsan. Sed lacinia egestas tortor, eget tristiqu dolor congue sit amet. Curabitur ut nisl a nisi consequat mollis sit amet quis nisl. Vestibulum hendrerit velit at odio sodales pretium. Nam quis tortor sed ante varius sodales. Etiam lacus arcu, placerat sed laoreet a, facilisis sed nunc. Nam gravida fringilla ligula, eu congue lorem feugiat eu.


Lorem ipsum dolor sit amet, consectetur adipiscing elit. Aenean aliquam lectus vel orci malesuada accumsan. Sed lacinia egestas tortor, eget tristiqu dolor congue sit amet. Curabitur ut nisl a nisi consequat mollis sit amet quis nisl. Vestibulum hendrerit velit at odio sodales pretium. Nam quis tortor sed ante varius sodales. Etiam lacus arcu, placerat sed laoreet a, facilisis sed nunc. Nam gravida fringilla ligula, eu congue lorem feugiat eu.


Lorem ipsum dolor sit amet, consectetur adipiscing elit. Aenean aliquam lectus vel orci malesuada accumsan. Sed lacinia egestas tortor, eget tristiqu dolor congue sit amet. Curabitur ut nisl a nisi consequat mollis sit amet quis nisl. Vestibulum hendrerit velit at odio sodales pretium. Nam quis tortor sed ante varius sodales. Etiam lacus arcu, placerat sed laoreet a, facilisis sed nunc. Nam gravida fringilla ligula, eu congue lorem feugiat eu.

We are now discussing the \textbf{Ultimate answer to all knowledge}.
This line is particularly important it also adds an index entry for \textit{Ultimate answer to all knowledge}.\index{Ultimate answer to all knowledge}

\subsection{Demo listings}

We can also include listings like the following:

% Inline Listing example
\lstset{language=make,caption=Application Makefile,label=lst:app-make}
\begin{lstlisting}
CSRCS = app.c
SRC_DIR =..
include $(SRC_DIR)/config/application.cfg
\end{lstlisting}

Listings can also be referenced. References don't have to include chapter/table/figure numbers... so we can have hyperlinks \labelref{like this}{lst:makefile-test}.

\subsection{Tables}

We can also have tables... like \labelindexref{Table}{table:reports}.

\begin{center}
\begin{table}[htb]
  \caption{Generated reports - associated Makefile targets and scripts}
  \begin{tabular}{l*{6}{c}r}
    Generated report & Makefile target & Script \\
    \hline
    Full Test Specification & full_spec & generate_all_spec.py  \\
    Test Report & test_report & generate_report.py  \\
    Requirements Coverage & requirements_coverage &
    generate_requirements_coverage.py   \\
    API Coverage & api_coverage & generate_api_coverage.py  \\
  \end{tabular}
  \label{table:reports}
\end{table}
\end{center}
