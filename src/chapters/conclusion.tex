\chapter{Conclusion}
\label{chapter:conclusion}

This is the conclusion.  ce ați obținut, ce obiective ați avut, cum este relevant proiectul, ce rezultate ați obținut, cum ați evaluat.

The Internet allows anyone to create content, publish and spread information. Social platforms are lowering the barrier of entry and are making it especially easy for everyone to have a voice on the Internet and as a result millions of messages and content of all forms is generated daily. Making sense of everything, and keeping track is becoming difficult and therefore it is time for tools that help understand the content to evolve and adapt to these mediums and make content exploration as easy as it is to post a message.
\newline
{\project}  attempts to solve this problem of content discovery and exploration. {\project}  endeavors to understand the content being published and presents it to the user in a way that is accessible and easy to use. It creates clusters of messages by interpreting content and presents it to the user in a web interface that allows for him to browse through a large number of messages efficiently.
\newline
The feature that makes {\project}  relevant for the fast passed rate of tweets is its ability to parse the messages in real time. The data is not based on an archived corpus of documents but on streaming tweets as they happen. This way popular events, news and messages get reported in the interface and the user is able to keep in touch with what is happening right now.
\newline
\begin{itemize}
	\item Getting real time data from Twitter using its API based on user queries.
	\item Parsing tweets as they arrive. Using part-of-speech tagging to make annotations that help filter messages and extract important information.
	\item Using a clustering algorithm that is able to group messages, that has the ability to configure precision and that can run in parallel for a choice between speed and precision. 
	\item Building a decoupled system that can easily scale through the use of queues which allow for different rates of consumption and multiple consumers that can process the workload in parallel.
	\item Presenting the information through an accessible medium: the web browser, with an easy to use interface that allows the information to be explored.
\end{itemize}